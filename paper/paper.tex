\documentclass[12pt,a4paper]{article}
\usepackage[utf8]{inputenc}
\usepackage{amsmath}
\usepackage{amsfonts}
\usepackage{amssymb}
\usepackage[romanian]{babel}
\author{ANDREI-ALIN CORODESCU}
\title{Prover pentru logica separatoare relațională în Maude}
\begin{document}
\begin{titlepage}
\begin{center}
\textbf{
UNIVERSITATEA "ALEXANDRU IOAN CUZA" DIN IAȘI
}
\\
\textbf{FACULTATEA DE INFORMATICĂ}
\end{center}
   \vspace{50mm}
\begin{center}
	\Large\textbf {TITLU PROVIZORIU}\\
	\vspace{30mm}
	\large\textit {ANDREI-ALIN CORODESCU}
	\\
	\vspace{20mm}
	\textbf{Sesiunea: }\textit{iulie, 2018}\\
	\vspace{30mm}
	\textbf{Coordonator științific}\\
	\textbf{\textit{Conf. Dr. Ciobâcă Ștefan}}
	\vspace{30mm}
\end{center}
\end{titlepage}
\tableofcontents

\section{Introducere}
De ce am ales aceasta tema?
\section{Fundamente Teoretice}
În acest capitol voi prezenta pe scurt fundamentele teoretice ale prezentei lucrări.
\subsection{Logica Hoare}
\subsection{Logica Separatoare}
\subsection{Logica Separatoare Relațională}
\subsection{Logica Rescrierii}
\section{Prover Logică Separatoare Relațională}
Ordinea subsectiunilor mai poate fi schimbata ulterior. Capitolul cel mai semnificativ din lucrare, va contine toate detaliile de implementare ale solutiei curente
\subsection{Prezentare Generală Maude}
\subsection{Modelarea Logicilor Separatoare}
\subsection{Interacțiunea prin LOOP-MODE}
\subsection{Schemă Generală de funcționare}
\subsection{Procese automatizate}
\subsection{Interfața}
\subsection{Viitor}
\section{Cercetări adiționale}
\subsection{Automatizarea demonstrațiilor}
\subsection{Extensii pentru programe concurente}
\subsection{Java+ITP}

\section{Concluzii}
\end{document}