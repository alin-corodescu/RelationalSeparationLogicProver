\documentclass[12pt,a4paper]{article}
\usepackage[utf8]{inputenc}
\usepackage[T1]{fontenc}
\usepackage{amsmath}
\usepackage{amsfonts}
\usepackage{amssymb}
\usepackage[english]{babel}
\usepackage{array}
\usepackage{tabularx}
\usepackage{ebproof}
\author{ANDREI-ALIN CORODESCU}
\title{Program Equivalence : Relational Separation Logic interactive prover implemented in Maude}
\newcommand{\hq}[4]{
	\left\{{#1}\right\}
	\let\scriptstyle\textstyle 
	\substack{{#2} \\ 	\let\scriptstyle\textstyle {#3}} 
	\left\{{#4}\right\}
}

\begin{document}
\begin{titlepage}
\begin{center}
\textbf{
"ALEXANDRU IOAN CUZA" UNIVERSITY OF IAȘI
}
\\
\textbf{FACULTY OF COMPUTER SCIENCE}
\end{center}
   \vspace{40mm}
\begin{center}
	\Large\textbf {Program Equivalence : Relational Separation Logic interactive prover implemented in Maude}\\
	\vspace{40mm}
	\large\textit {ANDREI-ALIN CORODESCU}
	\\
	\vspace{20mm}
	\textbf{Session: }\textit{July, 2018}\\
	\vspace{30mm}
	\textbf{Scientific Coordinator}\\
	\textbf{\textit{Conf. Dr. Ciobâcă Ștefan}}
	\vspace{30mm}
\end{center}
\end{titlepage}
\tableofcontents
\pagebreak

\section{Introduction}
The present paper describes the development of an interactive tool for reasoning how to programs are related, based on studied and previously used theoretical concepts and technologies which facilitate the implementation. \\

The tool represents an implementation of Hoare Logic - which allows formal reasoning about a program - , along  with 2 of its extensions, namely the Separation Logic (named Separation Logic from now on) and Relational Separation Logic \cite{relational} (named Relational Logic from now on). The 2 extensions simplify the Hoare Logic proofs, mainly using the "*" connector, allowing for local reasoning of effects of statements in a program . The tool has been implemented in Maude, a high performance logical framework with powerful metalanguage applications which facilitate the implementations of executable environments for logics.\\

The tool is built as a CLI which helps \cite{primer} \cite{SeparationLogic} \cite{JAVAITP} \cite{REWRITING} \cite{maudeprimer} \cite{manual} \cite{rewrConcurrency} \cite{cyclist} with reasoning how two programs are related using Relational Separation Logic specifications. As a consequence of the dependency of Relational Separation Logic on Separation Logic, proofs about single programs using the latter are also supported by the tool. The tool has been developed with extensibility in mind, the main desired extensions being concurrent programs support and automatic proofs. \\

The rest of the paper is organized as follows: 


\section{Contributions}
Personal contributions to the realization of the project : 
\begin{itemize}
	\item Modelled the Relational Logic and Separation Logic using Maude equational and rewriting logic specifications . 
	\item Developed an interactive tool for reasoning about program behaviour using the aforementioned logics.
	\item Automation of some tasks which makes the tool more convenient to use .
	\item Examples of formal proofs done using the tool
\end{itemize}
\section{Description of the problem}
The problem this project is aiming to solve is related to formal reasoning about the execution of code, mainly focusing on how two programs are related to each other (most often the relation to be proven is equivalence)\\

Comparing programs or code fragments and studying their equivalence is part of every software engineer's activities when they are testing an alternative implementation for an existing solution, fixing bugs, launching new product versions, etc . Naturally, for every process completed manually there are efforts being made in order to make it more efficient, less error-prone and, in the end, automate the process all together. Once such a task is automated in software engineering, it can be included in the flow of any research or development phase of a product. An example benefiting from a formal proof of program equivalence is compiler optimization, where the optimized code needs to be equivalent to the input one . \\

This project aims to lay the foundations of a tool which facilitates formal reasoning about program equivalence with a focus on extensibility and automation of tasks.

\section{Previous work}
Previous work related to the topic has been done mostly in terms of Separation Logic based tools, with Relational Separation Logic not being treated as much. \\

A notable example which also uses the same framework as this project, namely Maude, is the Java+ITP\cite{JAVAITP} tool, which enables analysis of Java programs using Separation Logic. The implementation relies heavily on Maude's ITP (iterative theorem prover).
\bibliographystyle{unsrt}
\bibliography{bibliography}


\end{document}