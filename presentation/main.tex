% Copyright 2004 by Till Tantau <tantau@users.sourceforge.net>.
%
% In principle, this file can be redistributed and/or modified under
% the terms of the GNU Public License, version 2.
%
% However, this file is supposed to be a template to be modified
% for your own needs. For this reason, if you use this file as a
% template and not specifically distribute it as part of a another
% package/program, I grant the extra permission to freely copy and
% modify this file as you see fit and even to delete this copyright
% notice. 



\documentclass{beamer}
\usepackage{ebproof}
\usepackage{listings}
\usepackage[utf8]{inputenc}
\usepackage{color}

\definecolor{codegreen}{rgb}{0,0.6,0}
\definecolor{codegray}{rgb}{0.5,0.5,0.5}
\definecolor{codepurple}{rgb}{0.58,0,0.82}
\definecolor{backcolour}{rgb}{0.95,0.95,0.92}
\lstdefinestyle{mystyle}{
	%backgroundcolor=\color{backcolour},   
	commentstyle=\color{codegreen},
	keywordstyle=\color{magenta},
	numberstyle=\tiny\color{codegray},
	stringstyle=\color{codepurple},
	basicstyle=\footnotesize,
	breakatwhitespace=false,         
	breaklines=true,                 
	captionpos=b,                    
	keepspaces=true,                 
	%numbers=left,                    
	%numbersep=5pt,                  
	showspaces=false,                
	showstringspaces=false,
	showtabs=false,                  
	tabsize=2
}
\lstset{style=mystyle}

\usepackage{tabularx}
% There are many different themes available for Beamer. A comprehensive
% list with examples is given here:
% http://deic.uab.es/~iblanes/beamer_gallery/index_by_theme.html
% You can uncomment the themes below if you would like to use a different
% one:
%\usetheme{AnnArbor}
%\usetheme{Antibes}
%\usetheme{Bergen}
%\usetheme{Berkeley}
%\usetheme{Berlin}
%\usetheme{Boadilla}
%\usetheme{boxes}
%\usetheme{CambridgeUS}
%\usetheme{Copenhagen}
%\usetheme{Darmstadt}
%\usetheme{default}
%\usetheme{Frankfurt}
%\usetheme{Goettingen}
%\usetheme{Hannover}
%\usetheme{Ilmenau}
%\usetheme{JuanLesPins}
%\usetheme{Luebeck}
\usetheme{Madrid}
%\usetheme{Malmoe}
%\usetheme{Marburg}
%\usetheme{Montpellier}
%\usetheme{PaloAlto}
%\usetheme{Pittsburgh}
%\usetheme{Rochester}
%\usetheme{Singapore}
%\usetheme{Szeged}
%\usetheme{Warsaw}

\newcommand{\rl}[2]{
	\let\scriptstyle\textstyle 
	\substack{{#1} \\ \\ 	\let\scriptstyle\textstyle {#2}} 
}
\newcommand{\hq}[4]{
	\left\{{#1}\right\}
	\rl{#2}{#3}
	\left\{{#4}\right\}
}


\usepackage{combelow}

\usepackage{mathtools}

\newcommand{\defeq}{\stackrel{\mathclap{\normalfont{def}}}{=}}


\title{Program Equivalence: An Interactive Relational Separation Logic Prover Implemented in Maude}


\author[Andrei-Alin Corodescu]{Andrei-Alin Corodescu \\ \vspace{1.5cm} Scientific Coordinator: \\ Conf. Dr. Ciobâcă Ștefan}
% - Give the names in the same order as the appear in the paper.
% - Use the \inst{?} command only if the authors have different
%   affiliation.
\institute[Alexandru Ioan Cuza University of Iași] % (optional, but mostly needed)

\date{July 4, 2018}
\subject{Theoretical Computer Science}



\defbeamertemplate*{footline}{mytheme}
{
	\leavevmode%
	\hbox{%
		\begin{beamercolorbox}[wd=.5\paperwidth,ht=2.25ex,dp=1ex,center]{author in head/foot}%
			\usebeamerfont{author in head/foot}\insertshortauthor~~(\insertshortinstitute)
		\end{beamercolorbox}%
		\begin{beamercolorbox}[wd=.5\paperwidth,ht=2.25ex,dp=1ex,right]{date in head/foot}%
			\usebeamerfont{date in head/foot}\insertshortdate{}\hspace*{2em}
			\insertframenumber{} / \inserttotalframenumber\hspace*{2ex} 
	\end{beamercolorbox}}%
	\vskip0pt%
}
\usebeamertemplate{mytheme}
% This is only inserted into the PDF information catalog. Can be left
% out. 

% If you have a file called "university-logo-filename.xxx", where xxx
% is a graphic format that can be processed by latex or pdflatex,
% resp., then you can add a logo as follows:

% \pgfdeclareimage[height=0.5cm]{university-logo}{university-logo-filename}
% \logo{\pgfuseimage{university-logo}}

% Delete this, if you do not want the table of contents to pop up at
% the beginning of each subsection:
\AtBeginSubsection[]
{
  \begin{frame}<beamer>{Outline}
    \tableofcontents[currentsection,currentsubsection]
  \end{frame}
}

% Let's get started
\begin{document}

\begin{frame}
  \titlepage
\end{frame}

\begin{frame}{Outline}
  \tableofcontents
  % You might wish to add the option [pausesections]
\end{frame}

% Section and subsections will appear in the presentation overview
% and table of contents.
\section{The Problem}

\begin{frame}{The Problem}
\begin{block}{Problem Description}
	\begin{itemize}
	\item Program Equivalence. 
	\item Absence of an implementation for Relational Separation Logic, theoretical framework supporting formal proofs for program equivalence.
	\end{itemize}
    \pause
\end{block}
	\begin{block}{Difficulties}
		\begin{itemize}
			\item Representing the theoretical concepts
			\item Computationally-hard problems
			\item User experience
		\end{itemize}
	\end{block}
\end{frame}

\begin{frame}[fragile]{Example}
\begin{center}
\begin{tabular*}{0.9\textwidth}{@{\extracolsep{\fill} } c  c }
	Program 1: \vspace{0.5cm} & Program 2: \vspace{0.5cm} \\
	\begin{lstlisting}
while (c != nil) do
	x := [y];
	[c] := -x;
	c := [c + 1]
od
	\end{lstlisting} &
	\begin{lstlisting}
x' := [y'];
while (c' != nil) do
	[c'] := -x';
	c' := [c' + 1]
od
	\end{lstlisting}

\end{tabular*}
	\begin{block}{Note}
	If the address \texttt{y} is part of the list starting at \texttt{c}, the two programs are \alert{not} equivalent.
\end{block}
\end{center}
\end{frame}

\section{Technologies and theoretical concepts}

\subsection{Relational Separation Logic}
\begin{frame}{Relational Separation Logic}
\begin{itemize}
	\item Helps reason about how two programs are related
	\item {Hoare Quadruples : \(\hq{R}{C}{C\prime}{S}\)}
	\item {Example proof rule (Consequence) : \begin{center}	\(
		\begin{prooftree}
		\Hypo{R \Rightarrow R_1}
		\Hypo{\hq{R_1}{C}{C\prime}{S_1}}
		\Hypo{S_1 \Rightarrow S}
		\Infer 3 {\hq{R}{C}{C\prime}{S}}
		\end{prooftree}	
		\)
		\end{center}
	}
	\item Axioms : \(
	\{E \mapsto -\} [E] := F \{E \mapsto F\}.
	\)
\end{itemize}
\end{frame}
\begin{frame}[fragile]{Relational Separation Logic - Example}
Consider the two programs exemplified earlier to be denoted by \(C_1\) and \(C_2\).
\begin{example}
	
	\begin{itemize}
		\item
		\textsf{List x \(\defeq\) (x = nil \(\land\) Emp) \(\lor \) 
			\(\exists na.\) 
			\(\left(\rl{x \mapsto a,n}{x \mapsto a,n}\right) *\) List\(\ n\) 
		}
		\item \(\left\{\left(Same * List\ c * \left(\rl{y \mapsto x_0}{y\prime \mapsto x_0}\right)\right) \land y = y\prime \land c = c\prime \right\}\)
		\\
		\[\rl{C_1}{C_2}\]
		\(\left\{Same \land y = y\prime \land c = c\prime\right\}\)
	\end{itemize}
\end{example}
\end{frame}


\subsection{Maude}
\begin{frame}[fragile]{Maude}
\begin{itemize}
	\item Based on Rewriting logic
	\item {Natural Representation of logics through sorts and operators}
	\item Powerful meta language applications
\end{itemize}
\begin{example}
	\begin{lstlisting}
rl [Consequence] : { R } C1 -- C2  { S } => ((R => R1) <> ({R1} C1 -- C2 {S1})) <> (S1 => S) [nonexec] .
\end{lstlisting}
\end{example}
\end{frame}
\section{Demonstration}
\begin{frame}{Demo}
\end{frame}
% Placing a * after \section means it will not show in the
% outline or table of contents.
\section*{Conclusions}

\begin{frame}{Conclusions}
  \begin{itemize}
  \item The prover showcases:
  	\begin{itemize}
 		  		\item A promising executable environment for Relational Separation Logic which can be improved upon in the future
  		\item The features of Maude that make it fit for this purpose
  	\end{itemize} 
  \item Personal conclusions:
  \begin{itemize}
  \item Learning by modelling and applying logics
  \item Shortcomings of Maude because of it not being widely adopted
  \end{itemize}
  \end{itemize}
 
\end{frame}
\end{document}


