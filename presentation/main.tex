% Copyright 2004 by Till Tantau <tantau@users.sourceforge.net>.
%
% In principle, this file can be redistributed and/or modified under
% the terms of the GNU Public License, version 2.
%
% However, this file is supposed to be a template to be modified
% for your own needs. For this reason, if you use this file as a
% template and not specifically distribute it as part of a another
% package/program, I grant the extra permission to freely copy and
% modify this file as you see fit and even to delete this copyright
% notice. 



\documentclass{beamer}
\usepackage{ebproof}
\usepackage{listings}

\usepackage{color}

\definecolor{codegreen}{rgb}{0,0.6,0}
\definecolor{codegray}{rgb}{0.5,0.5,0.5}
\definecolor{codepurple}{rgb}{0.58,0,0.82}
\definecolor{backcolour}{rgb}{0.95,0.95,0.92}
\lstdefinestyle{mystyle}{
	%backgroundcolor=\color{backcolour},   
	commentstyle=\color{codegreen},
	keywordstyle=\color{magenta},
	numberstyle=\tiny\color{codegray},
	stringstyle=\color{codepurple},
	basicstyle=\footnotesize,
	breakatwhitespace=false,         
	breaklines=true,                 
	captionpos=b,                    
	keepspaces=true,                 
	%numbers=left,                    
	%numbersep=5pt,                  
	showspaces=false,                
	showstringspaces=false,
	showtabs=false,                  
	tabsize=2
}
\lstset{style=mystyle}
% There are many different themes available for Beamer. A comprehensive
% list with examples is given here:
% http://deic.uab.es/~iblanes/beamer_gallery/index_by_theme.html
% You can uncomment the themes below if you would like to use a different
% one:
%\usetheme{AnnArbor}
%\usetheme{Antibes}
%\usetheme{Bergen}
%\usetheme{Berkeley}
%\usetheme{Berlin}
%\usetheme{Boadilla}
%\usetheme{boxes}
%\usetheme{CambridgeUS}
%\usetheme{Copenhagen}
%\usetheme{Darmstadt}
%\usetheme{default}
%\usetheme{Frankfurt}
%\usetheme{Goettingen}
%\usetheme{Hannover}
%\usetheme{Ilmenau}
%\usetheme{JuanLesPins}
%\usetheme{Luebeck}
\usetheme{Madrid}
%\usetheme{Malmoe}
%\usetheme{Marburg}
%\usetheme{Montpellier}
%\usetheme{PaloAlto}
%\usetheme{Pittsburgh}
%\usetheme{Rochester}
%\usetheme{Singapore}
%\usetheme{Szeged}
%\usetheme{Warsaw}

\newcommand{\hq}[4]{
	\left\{{#1}\right\}
	\let\scriptstyle\textstyle 
	\substack{{#2} \\ \\ 	\let\scriptstyle\textstyle {#3}} 
	\left\{{#4}\right\}
}


\title{Program Equivalence: An Interactive Relational Separation Logic Prover Implemented in Maude}


\author{Andrei Alin Corodescu \\ \vspace{1.5cm} Scientific Coordinator: \\ Conf. Dr. Ciobaca Stefan}
% - Give the names in the same order as the appear in the paper.
% - Use the \inst{?} command only if the authors have different
%   affiliation.
\institute[Alexandru Ioan Cuza University of Iasi] % (optional, but mostly needed)


\subject{Theoretical Computer Science}
% This is only inserted into the PDF information catalog. Can be left
% out. 

% If you have a file called "university-logo-filename.xxx", where xxx
% is a graphic format that can be processed by latex or pdflatex,
% resp., then you can add a logo as follows:

% \pgfdeclareimage[height=0.5cm]{university-logo}{university-logo-filename}
% \logo{\pgfuseimage{university-logo}}

% Delete this, if you do not want the table of contents to pop up at
% the beginning of each subsection:
\AtBeginSubsection[]
{
  \begin{frame}<beamer>{Outline}
    \tableofcontents[currentsection,currentsubsection]
  \end{frame}
}

% Let's get started
\begin{document}

\begin{frame}
  \titlepage
\end{frame}

\begin{frame}{Outline}
  \tableofcontents
  % You might wish to add the option [pausesections]
\end{frame}

% Section and subsections will appear in the presentation overview
% and table of contents.
\section{The Problem}

\begin{frame}{The Problem}
\begin{block}{Problem Description}
    Reducing the gap between theoretical and practical aspects of formal program equivalence verification to increase software quality by making robust methods of verification accessible and easy to use.
    \pause
\end{block}
	\begin{block}{Difficulties}
		\begin{itemize}
			\item Representing the theoretical concepts
			\item Computationally-hard problems
			\item User experience
		\end{itemize}
	\end{block}
\end{frame}

\section{Technologies and theoretical concepts}

\subsection{Relational Separation Logic}
\begin{frame}{Relational Separation Logic}
\begin{itemize}
	\item Helps reason about how two programs are related
	\item {Hoare Quadruples : \(\hq{R}{C}{C\prime}{S}\)}
	\item {Proof rules : 	\(
		\begin{prooftree}
		\Hypo{R \Rightarrow R_1}
		\Hypo{\hq{R_1}{C}{C\prime}{S_1}}
		\Hypo{S_1 \Rightarrow S}
		\Infer 3 {\hq{R}{C}{C\prime}{S}}
		\end{prooftree}	
		\)
	}
	\item Separation Logic axioms : \(
	\{E \mapsto -\} [E] := F \{E \mapsto F\}.
	\)
\end{itemize}
\end{frame}

\subsection{Maude}
\begin{frame}[fragile]{Maude}
\begin{itemize}
	\item Based on Rewriting logic
	\item {Natural Representation of logics through sorts and operators}
	\item Powerful meta language applications
\end{itemize}
\begin{example}
	\begin{lstlisting}
rl [Consequence] : { R } C1 -- C2  { S } => ((R => R1) <> ({R1} C1 -- C2 {S1})) <> (S1 => S) [nonexec] .
\end{lstlisting}
\end{example}
\end{frame}

\section{Prover Implementation}
\subsection{Goals}
\begin{frame}{Goals}
\begin{itemize}
	\item The central concept of the prover is \alert{Goal}, which represents something to be proven.
	\item A \alert{Goal} is consumed if it is an matched with an \textit{axiom}, is \textit{manually admitted} by the user, is \textit{automatically proven} (in case of implications) or it is \textit{replaced} by the goals generated by applying a proof rule. 
	\item Goals are stored in a \alert{GoalStack} structure
\end{itemize}
\end{frame}

\subsection{Axiom Recognition}
\begin{frame}{Axiom Recognition}
\begin{itemize}
	\item Simple goals that match \alert{axioms} are automatically admitted by the prover.
	\item Equality between variables is \alert{interpreted} before matching axioms
	\item Axioms are represented as terms with variables
\end{itemize}
\end{frame}

\subsection{Automatic Proof of Implications}
\begin{frame}{Automatic Proof of Implications}
\begin{itemize}
	\item Uses the \alert{search} functionality of Maude
	\item Searches for a series of \alert{rewrites} from \texttt{R => S} to \texttt{true}  
	\item Rewrite rules denoting relation equivalences and implications
\end{itemize}
\end{frame}

\begin{frame}{Demo}
\end{frame}
% Placing a * after \section means it will not show in the
% outline or table of contents.
\section*{Conclusions}

\begin{frame}{Conclusions}
  \begin{itemize}
  \item The prover showcases:
  	\begin{itemize}
 		  		\item A promising executable environment for Relational Separation Logic which can be improved upon in the future
  		\item The features of Maude that make it fit for this purpose
  	\end{itemize} 
  \item Personal conclusions:
  \begin{itemize}
  \item Learning by modelling and applying logics
  \item Shortcomings of Maude because of it not being widely adopted
  \end{itemize}
  \end{itemize}
 
\end{frame}
\end{document}


